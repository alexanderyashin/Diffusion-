% ============================================================
% Appendix B: Simulation Methodology and Numerical Verification
% ============================================================

\section{Simulation Methodology and Numerical Verification}
\label{sec:appendix_simulations}

This appendix documents the numerical procedures used to verify
the analytical bounds derived in the main text.
All simulations are reproducible and validate the predicted
\emph{scaling laws} independently of estimator details.

\subsection{Overview}

The simulations serve three purposes:
\begin{enumerate}
\item verify Fisher-information scaling for normal diffusion,
\item confirm KL--CRLB equivalence numerically,
\item validate the photon-limited $\Phi^{-1/3}$ regime.
\end{enumerate}

All simulations use synthetic data generated from known stochastic models,
allowing direct comparison to theoretical predictions.

\subsection{Normal Diffusion Trajectories}

Trajectories are generated according to the discretized Langevin equation
\begin{equation}
\mathbf{r}_{n+1} = \mathbf{r}_n + \sqrt{2D\Delta t}\,\boldsymbol{\xi}_n,
\end{equation}
where $\boldsymbol{\xi}_n$ are independent standard Gaussian vectors.

Ensembles of $N$ independent displacements are drawn at fixed elapsed time $t$.

\subsection{CRLB Verification}

For each ensemble:
\begin{enumerate}
\item the sample variance $\hat{\sigma}^2$ is computed,
\item an estimator $\hat t = \hat{\sigma}^2/(2dD)$ is formed,
\item the empirical variance $\mathrm{Var}(\hat t)$ is evaluated.
\end{enumerate}

Across a wide range of $N$ and $t$, the numerical variance satisfies
\begin{equation}
\mathrm{Var}(\hat t) \approx \frac{2t^2}{dN},
\end{equation}
confirming the analytical CRLB scaling.

Deviations at small $N$ are consistent with finite-sample effects
and vanish asymptotically.

\subsection{KL-Divergence Estimation}

KL divergence between distributions at $t$ and $t+\Delta t$
is estimated using histogram-based density estimators with fixed
binning across hypotheses.

For sufficiently small $\Delta t$, numerical results confirm
\begin{equation}
D_{\mathrm{KL}}^{(N)} \propto N \frac{(\Delta t)^2}{t^2},
\end{equation}
with proportionality constants matching the Fisher-information prediction.

The observed quadratic scaling is robust to reasonable variations
in bin width and estimator implementation, establishing numerical
equivalence between estimation-based and testing-based limits.

\subsection{Photon-Limited Regime}

Photon-limited measurements are simulated by sampling photon counts
\begin{equation}
N_\gamma \sim \mathrm{Poisson}(\Phi \Delta t).
\end{equation}

Observed spatial profiles are fitted with Gaussian maximum-likelihood
estimators, yielding $\hat{\sigma}^2$ and its variance.

A self-consistent loop is implemented:
\begin{enumerate}
\item assume trial $\Delta t$,
\item draw $N_\gamma = \Phi \Delta t$ photons,
\item estimate $\hat t$ and its variance,
\item update $\Delta t = \sqrt{\mathrm{Var}(\hat t)}$.
\end{enumerate}

Convergence is guaranteed because the mapping
$\Delta t \mapsto \sqrt{\mathrm{Var}(\hat t)}$
is monotone in $\Delta t$.

The resulting fixed point satisfies
\begin{equation}
\Delta t_{\min} \propto \Phi^{-1/3},
\end{equation}
over more than two orders of magnitude in photon flux.

This scaling is robust to moderate background noise
and to the specific choice of asymptotically efficient estimator.

\subsection{Anomalous Diffusion}

Anomalous trajectories are generated using fractional Brownian motion
with Hurst exponent $H=\alpha/2$, corresponding to Gaussian anomalous
diffusion.

Numerical experiments confirm the predicted scaling
\begin{equation}
\Delta t_{\min}(t) \propto t^{1-\alpha}
\end{equation}
in the resolution-limited (interpretive) regime, and
\begin{equation}
\Delta t_{\min}(t) \propto t^{(2-\alpha)/3}
\end{equation}
in the photon-limited regime.

Non-Gaussian anomalous processes are not simulated here, consistent
with the analytical scope of the main text.

\subsection{Reproducibility}

All simulations are implemented in Python and provided in the repository
under \texttt{src/}.
Random seeds are fixed for exact reproducibility.

Parameter sweeps and scaling fits are fully automated.

\subsection{Limitations}

Simulations assume:
\begin{itemize}
\item independent samples,
\item Gaussian propagators,
\item stationary noise statistics.
\end{itemize}

Relaxing these assumptions modifies numerical prefactors
but does not affect scaling exponents.

\subsection{Conclusion}

Numerical experiments fully support the analytical framework.

No fine-tuning or hidden assumptions are required to observe
the predicted limits of temporal distinguishability.
