% ============================================================
% 08_discussion_limits.tex
% ============================================================

\section{Discussion: Assumptions, Limits, and Scope}
\label{sec:discussion}

This section critically examines the assumptions underlying the framework,
clarifies the role of numerical constants, and delineates the precise domain
of validity. Its purpose is to pre-empt common objections and to make explicit
what the theory does and does not claim.

\subsection{On the Role of Numerical Constants}

Throughout the paper, scaling relations are emphasized over exact prefactors.
This is intentional.

\paragraph{Statistical thresholds.}
Both the Cramér--Rao bound and the KL-divergence criterion define limits
conditional on a chosen confidence level or hypothesis-testing threshold.
Once such a threshold is fixed, all numerical constants are fully specified.
Different choices alter only prefactors, not scaling laws.

\paragraph{Universality.}
The exponents $1/2$, $1/3$, $1-\alpha$, and $(2-\alpha)/3$ are invariant
under estimator choice, confidence level, and experimental implementation
within the stated model classes. This invariance is the physically meaningful
content of the theory.

\subsection{Estimator Dependence and Optimality}

The CRLB represents a lower bound achievable only by asymptotically efficient
estimators.

\begin{itemize}
\item Suboptimal estimators may yield larger $\Delta t_{\min}$,
\item No estimator can improve the predicted scalings without violating
information-theoretic bounds under the stated assumptions.
\end{itemize}

Thus, experimental deviations toward \emph{worse} resolution do not refute
the theory, while systematic improvements beyond the bound do.

\subsection{Sample Independence and Correlations}

The derivations assume independent samples.
In practice, correlations arise from:

\begin{itemize}
\item Motion blur due to finite exposure time,
\item Trajectory oversampling,
\item Viscoelastic memory effects.
\end{itemize}

These effects reduce the effective sample size $N_{\mathrm{eff}} < N$.
Replacing $N$ by $N_{\mathrm{eff}}$ restores the validity of the bounds.
Failure to account for correlations reflects an experimental limitation,
not a theoretical loophole.

\subsection{Gaussianity and Non-Gaussian Extensions}

The analytical Fisher-information results rely on Gaussian propagators.
This assumption holds for:

\begin{itemize}
\item Free Brownian motion,
\item Fractional Brownian motion (locally Gaussian increments).
\end{itemize}

For non-Gaussian processes (e.g., CTRW with power-law waiting times),
closed-form Fisher-information expressions may not exist.
In such cases:

\begin{itemize}
\item KL-divergence remains well-defined and computable numerically,
\item Monte Carlo estimation of distinguishability is sufficient,
\item The framework remains operationally valid.
\end{itemize}

\subsection{Why This Is Not Trivial Experimental Resolution}

A frequent objection is that $\Delta t_{\min}$ merely reflects experimental
limitations.

This interpretation is incorrect within the stated model classes.

\begin{enumerate}
\item The bounds are derived from \emph{probability distributions}, not from
hardware specifications.
\item Even idealized detectors cannot surpass the limits without access to
additional information beyond the spatial statistics considered here.
\item The photon-limited $\Phi^{-1/3}$ scaling is nontrivial and counterintuitive,
      differing fundamentally from standard $1/\sqrt{N}$ noise reduction.
\end{enumerate}

The limits are therefore fundamental to inference, not to instrumentation.

\subsection{Relation to Other Time Scales}

The present $\Delta t_{\min}$ must be distinguished from:

\begin{itemize}
\item Nyquist sampling limits (signal discretization),
\item First-passage times (event statistics),
\item Quantum uncertainty bounds on clocks.
\end{itemize}

Here, time is neither a sampling interval nor a physical observable,
but a parameter inferred from spatial probability distributions.

\subsection{Scope and Non-Claims}

The theory does \emph{not} claim:

\begin{itemize}
\item The existence of a fundamental quantum of time,
\item Modifications to diffusion dynamics,
\item Violations of classical or quantum mechanics.
\end{itemize}

It \emph{does} claim:

\begin{itemize}
\item A fundamental information-theoretic limit on temporal inference,
\item Universality across normal and anomalous diffusion,
\item Experimental decidability with current technology.
\end{itemize}

\subsection{Position Within the Ontology of Continua}

Within the Ontology of Continua framework, this result can be interpreted as a
minimal physical realization of level $K_2$: time as an operationally emergent
dimension defined by distinguishability.

Importantly, the present results stand fully independently of that broader
framework and require no ontological or metaphysical assumptions.

\subsection{Summary}

All assumptions are explicit, all limits are quantitative,
and all predictions are falsifiable.
No free parameters or interpretive ambiguities remain at the level of scaling laws.
