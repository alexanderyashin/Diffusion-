% ============================================================
% 00_abstract.tex
% ============================================================

\begin{abstract}

We establish information-theoretic limits on temporal distinguishability
in diffusive and anomalously diffusive systems.

Time is treated not as a primitive variable but as an inferred parameter
of stochastic dynamics, constrained by the distinguishability of
probability distributions under finite spatial resolution, finite
statistics, and measurement noise.

For normal diffusion, Fisher information yields an estimator-independent
lower bound on temporal resolution,
$\Delta t_{\min}(t) \sim t\sqrt{2/(dN)}$,
up to dimensionless constants.
An equivalent local bound follows from hypothesis testing via the
Kullback--Leibler divergence, demonstrating consistency between estimation
and testing frameworks in the small-$\Delta t$ regime.

In the photon-limited regime, a self-consistent analysis predicts a
nontrivial temporal resolution scaling
$\Delta t_{\min} \propto \Phi^{-1/3}$,
distinct from the standard $1/\sqrt{N}$ noise suppression.
This scaling is derived analytically, verified in numerical simulations,
and is directly testable with existing single-particle tracking and
fluorescence-based measurement techniques.

The framework is extended to anomalous diffusion, revealing qualitatively
distinct temporal regimes governed by the anomalous exponent and the
associated scaling of distinguishability with time.
All results follow from first principles and are falsifiable through
explicit experimental protocols.

The present work provides a minimal operational definition of time
resolution in stochastic systems and a general template for analyzing
temporal distinguishability under information constraints.

\end{abstract}
